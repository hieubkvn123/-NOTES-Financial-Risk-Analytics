\section{Caratheodory Extension Theorem}
\subsection{Semi-ring of sets}
\begin{definition}[Semi-ring of sets]
    Given a set $X$, a collection of subsets of $X$ - $\mathcal{A}$ is called a semi-ring of subsets in $X$ if it satisfies the following conditions:
    \begin{itemize}
        \item $\emptyset \in \mathcal{A}$.
        \item \textbf{Closure under intersection} : $A, B \in \mathcal{A} \implies A \cap B \in \mathcal{A}$.
        \item If $A, B \in \mathcal{A}$, there exists a \textbf{finite disjoint} collection of subsets $\{I_n\}_{n=1}^N\subset \mathcal{A}$ such that:
        \begin{align*}
            A \setminus B = \bigcup_{n=1}^N I_n
        \end{align*}
    \end{itemize}
\end{definition}

\noindent \textbf{Remark} : Notice that any semi-ring of sets over a set $X$ is also a $\pi$-system over $X$.

\subsection{$\sigma$-finiteness of measure}
\begin{definition}[$\sigma$-finite measure]
    Let $(X, \mathcal{A})$ be a measurable space and $\mu:\mathcal{A} \to [0, \infty]$ be a measure (or pre-measure) defined on it. Then $\mu$ is called $\sigma$-finite if $X$ can be \textbf{covered by countably many measurable sets with finite measure}. In other words,
    There exists $\{S_n\}_{n=1}^\infty \subset \mathcal{A}$, $\mu(S_n) < \infty$ such that $X = \bigcup_{n=1}^\infty S_n$.
\end{definition}

\subsection{Theorem and proof}
The Caratheodory Extension Theorem involves the uniqueness and existence of extension of pre-measures. Before diving into the theorem, we look at the following lemma \ref{lem:uniqueness_of_extension_lemma}, which will help us prove the uniqueness.
\begin{proposition}{Uniqueness of measure}{uniqueness_of_extension_lemma}
    Suppose that $\mu_1, \mu_2$ are measures on $(X, \mathcal{E})$ such that $\mu_1(X)=\mu_2(X)<\infty$. Let $\mathcal{A}\subset\mathcal{E}$ be a $\pi$-system over $X$ such that $\mathcal{E}=\sigma(\mathcal{A})$. Then,
    \begin{align*}
        \mu_1\Big|_{\mathcal{A}} := \mu_2\Big|_{\mathcal{A}} \implies \mu_1 := \mu_2
    \end{align*}

    In other words, it two measures agree on a $\pi$-system, they also agree on the $\sigma$-algebra generated by that $\pi$-system.
\end{proposition}

\begin{proof*}[Proposition \ref{prop:uniqueness_of_extension_lemma}]
    Let $\mathcal{D}$ be the set where $\mu_1, \mu_2$ agrees:
    \begin{align*}
        \mathcal{D} = \Big\{ A \in \mathcal{E} : \mu_1(A) = \mu_2(A) \Big\}
    \end{align*}

    \noindent\newline Hence, we have $\mathcal{A}\subseteq \mathcal{D}$.

    \begin{subproof}{\newline Claim : $\mathcal{D}$ is a $\lambda$-system}
        \begin{itemize}
            \item $E \in \mathcal{D}$ by assumption.
            \item If $A, B\in\mathcal{D}$ and $A \subseteq B$. Since $B = (B\setminus A) \cup A$, we have:
            \begin{align*}
                \mu_1(B) &= \mu_1(A) + \mu_1(B \setminus A) \\
                \mu_2(B) &= \mu_2(A) + \mu_2(B \setminus A) \\
            \end{align*}
            \noindent But we have $\mu_1(A)=\mu_2(A)$ and $\mu_1(B)=\mu_2(B)$. Hence, $\mu_1(B\setminus A) = \mu_2(B\setminus A)$ and $B\setminus A \in \mathcal{D}$.

            \item Let $\{A_n\}_{n=1}^\infty$ be a countable disjoint sets in $\mathcal{D}$. Since $\mu_1(A_n)=\mu_2(A_n)$ for all $n\ge 1$. Hence:
            \begin{align*}
                \sum_{n=1}^\infty \mu_1(A_n) = \sum_{n=1}^\infty \mu_2(A_n) \implies \mu_1\Bigg( \bigcup_{n=1}^\infty A_n \Bigg) = \mu_2\Bigg( \bigcup_{n=1}^\infty A_n \Bigg)
            \end{align*}

            \noindent Therefore, we have $\bigcup_{n=1}^\infty A_n \in \mathcal{D}$.
        \end{itemize}
    \end{subproof}

    \noindent\newline Now that we have proved that $\mathcal{D}$ is a $\lambda$-system that contains a $\pi$-system, by Dynkin's $\pi$-$\lambda$ theorem \ref{thm:dynkin}, we have $\mathcal{A}\subseteq\sigma(\mathcal{A})\subseteq\mathcal{D}$. Therefore:
    \begin{align*}
        \mu_1\Big|_{\sigma(\mathcal{A})} := \mu_2\Big|_{\sigma(\mathcal{A})} \text{ or } \mu_1\Big|_{\mathcal{E}} := \mu_2\Big|_{\mathcal{E}}
    \end{align*}
\end{proof*}

\begin{theorem}{Caratheodory Extension Theorem}{caratheodory_ext}
    Let $X$ be a set and $\mathcal{A}$ be a semi-ring of sets over $X$. Let $\mu : \mathcal{A} \to [0,\infty]$ be a pre-measure defined on the semi-ring of sets. Then,
    \begin{itemize}
        \item There exists an extension of $\mu$, $\tilde\mu:\sigma(\mathcal{A}) \to [0, \infty]$, which is a measure on the $\sigma$-algebra generated by the semi-ring.
        \item If $\mu$ is $\sigma$-finite, then $\tilde\mu$ is unique.
    \end{itemize}
\end{theorem}

\begin{proof*}[Theorem \ref{thm:caratheodory_ext}]
    We have to prove both the existence and uniqueness of $\tilde\mu$.\newline 
    \begin{subproof}{$(i)$ Existence of $\tilde\mu$}
        We start by defining the outer measure $\mu^*:\mathcal{P}(X)\to [0,\infty]$ of the pre-measure as followed:
        \begin{align*}
            \mu^*(E) = \inf\Bigg\{ 
                \sum_{n=1}^\infty \mu(E_n) : E_n \in \mathcal{A}, E \subseteq \bigcup_{n=1}^\infty E_n
            \Bigg\}
        \end{align*}

        \noindent Now we restrict $\mu^*$ to the set of Caratheodory-measurable subsets only:
        \begin{align*}
            \mathcal{M} = \Bigg\{ E \subseteq X : \mu^*(A) = \mu^*(A\cap E) + \mu^*(A\cap E^c) , \ \forall A \subseteq X \Bigg\}
        \end{align*}

        \noindent \newline The strategy is to prove the following:
        \begin{itemize}
            \item $\mathcal{M}$ is a $\sigma$-algebra $\implies$ $\mu^*$ is a measure on $\mathcal{M}$.
            \item $\mathcal{A}\subset \mathcal{M} \implies \sigma(\mathcal{A}) \subset \mathcal{M}$.
            \item Finally, conclude that $\mu^*:\sigma(\mathcal{A})\to[0,\infty]$ is an extension of $\mu$ to $\sigma(\mathcal{A})$.
        \end{itemize}

        \begin{subproof}{\newline Claim 1 : $\mathcal{M}$ is a $\sigma$-algebra (In other words, $\mu^*$ is indeed a measure on $\mathcal{M}$)}
            \noindent It is trivial to prove closure under complement because of the symmetry in Caratheodory criterion. Hence, we will focus on proving closure under countable union.\newline 
            
            \noindent Let $\{E_n\}_{n=1}^\infty\subset \mathcal{M}$, we have to prove that $E = \bigcup_{n=1}^\infty E_n \in \mathcal{M}$. To do this, we use the same technique that we used to prove proposition \ref{prop:m_is_sigma_algebra} for Lebesgue measurable subsets. We make use of the following lemmas for the proof:
            \begin{itemize}
                \item \textbf{Proposition \ref{prop:disjoint_union_in_algebra}} : In an algebra $\mathcal{M}$, for any countable collection $\{E_n\}_{n=1}^\infty$, there exists a countable \textit{disjoint} collection $\{F_n\}_{n=1}^\infty$ such that $\bigcup_{n=1}^\infty E_n = \bigcup_{n=1}^\infty F_n$.
                \item \textbf{Proposition \ref{prop:intersection_with_measurable_collection}} : For any finite disjoint collection of measurable sets $\{E_n\}_{n=1}^N$:
                \begin{align*}
                    \mu^*\Bigg( A \cap \bigcup_{n=1}^N E_n \Bigg) = \sum_{n=1}^N \mu^*(A\cap E_n)
                \end{align*}
            \end{itemize}
        \end{subproof}

        \begin{subproof}{\newline Claim 2 : $\mathcal{A} \subset \mathcal{M}$}
           For any $E \in \mathcal{A}$, we have to show that for all $A\subseteq X$, we have:
           \begin{align*}
               \mu^*(A) = \mu^*(A \cap E) + \mu^*(A\cap E^c) 
           \end{align*}

           \noindent\newline By the definition of the outer measure $\mu^*$, for all $\epsilon > 0$, we can always find a countable collection $\{A_n\}_{n=1}^\infty\subset\mathcal{A}$ such that:
           \begin{align*}
               A \subset \bigcup_{n=1}^\infty A_n \text{ and } \mu^*(A) + \epsilon \ge \sum_{n=1}^\infty \mu^*(A_n)
           \end{align*}

           \noindent\newline Then, we also have:
           \begin{align*}
               A \cap E &\subseteq \bigcup_{n=1}^\infty (A_n \cap E) \\
               A \cap E^c &\subseteq \bigcup_{n=1}^\infty (A_n \cap E^c)
           \end{align*}

           \noindent\newline Hence, we have:
           \begin{align*}
               \mu^*(A\cap E) + \mu^*(A\cap E^c) &\le \mu^*\Bigg( \bigcup_{n=1}^\infty (A_n \cap E) \Bigg) + \mu^*\Bigg( \bigcup_{n=1}^\infty (A_n \cap E^c) \Bigg) \\
               &= \mu^*\Bigg( \bigcup_{n=1}^\infty (A_n \cap E) \Bigg) + \mu^*\Bigg( \bigcup_{n=1}^\infty (A_n \setminus E) \Bigg)
           \end{align*}

           \noindent\newline Since we have $A_n\cap E \in \mathcal{A}$ for all $n\ge 1$, we have:
           \begin{align*}
               \mu^*\Bigg( \bigcup_{n=1}^\infty (A_n \cap E) \Bigg) = \sum_{n=1}^\infty \mu^*(A_n\cap E)
           \end{align*}

           \noindent\newline Furthermore, we can always write $A_n\setminus E$ as a finite disjoint union of elements in $\mathcal{A}$, we also have:
           \begin{align*}
               \mu^*\Bigg( \bigcup_{n=1}^\infty (A_n \setminus E) \Bigg) = \sum_{n=1}^\infty \mu^*(A_n\setminus E)
           \end{align*}

           \noindent\newline Therefore, we can rewrite the above inequality as:
           \begin{align*}
               \mu^*(A\cap E) + \mu^*(A\cap E^c) &\le \sum_{n=1}^\infty \mu^*(A_n\cap E) + \sum_{n=1}^\infty \mu^*(A_n\setminus E) \\
               &= \sum_{n=1}^\infty \Big( \mu^*(A_n\cap E) + \mu^*(A_n\cap E^c) \Big) \\
               &= \sum_{n=1}^\infty \mu^*(A_n) \\
               &\le \mu^*(A) + \epsilon
           \end{align*}

           \noindent Taking $\epsilon\to 0$, we have $\mu^*(A) = \mu^*(A\cap E) + \mu^*(A\cap E^c)$. Hence, $E$ is measurable and we conclude that $\mathcal{A}\subset\mathcal{M}$.
        \end{subproof}

        \begin{subproof}{\newline Claim 3 : $\sigma(\mathcal{A}) \subset \mathcal{M}$}
            \newline This is a direct consequence of \textbf{Claim 2} due to the Dynkin's $\pi$-$\lambda$ theorem \ref{thm:dynkin}.
        \end{subproof}

        \noindent\newline With \textbf{Claim 1}, \textbf{Claim 2} and \textbf{Claim 3}, we define $\tilde\mu$ as a measure such that $\tilde\mu\Big|_{\sigma(\mathcal{A})}:=\mu^*\Big|_{\sigma(\mathcal{A})}$ and conclude the proof for existence of an extension measure.
    \end{subproof}

    \begin{subproof}{\newline (ii) Uniqueness of $\tilde\mu$}
        Suppose that $\mu_1$ and $\mu_2$ are two measures defined on $(X, \sigma(\mathcal{A})$ such that $\mu_1\Big|_{\mathcal{A}} := \mu_2\Big|_{\mathcal{A}}$. By proposition \ref{prop:uniqueness_of_extension_lemma}, since $\mu_1, \mu_2$ agrees on a $\pi$-system and $\mu_1(X)=\mu_2(X)<\infty$, $\mu_1$ and $\mu_2$ are identical and thus conclude the proof for uniqueness.
    \end{subproof}
\end{proof*}
