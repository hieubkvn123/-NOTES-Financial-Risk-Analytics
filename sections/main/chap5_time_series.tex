\newpage
\section{Time-series for financial data}
\subsection{Autoregressive Moving Average}
\subsubsection{$MA$ and $AR$ models}
\begin{definition}[White Noise]
    A \textbf{White noise sequence} $(Z_n)_{n\in\mathbb{Z}}$ is a sequence of i.i.d centered, unit variance random variables with
    \begin{align*}
        \mathbb{E}[Z_n] = 0, \ Cov(Z_n, Z_m) = \1{n=m}
    \end{align*}
\end{definition}

\begin{definition}[Moving Average (MA) model]
    In the $MA(q)$ model of order $q\ge 1$, the current state of the system is expressed as the \textbf{linear (independent) combination}:
    \begin{align*}
        \boxed{
            X_n = Z_n + \sum_{k=1}^q \beta_k Z_{n-k}
        }
    \end{align*}
    \noindent of $q$ previous states $Z_{n-1}, Z_{n-2}, \dots, Z_{n-q}$. $\beta_1, \dots, \beta_q$ is a sequence of deterministic coefficients. Define the lag operator $L$ as followed:
    \begin{align*}
        LZ_n = Z_{n-1}, \ L^kZ_n = Z_{n-k}
    \end{align*}
    \noindent We can then rewrite the $MA(q)$ model as:
    \begin{align*}
        \boxed{
            X_n = Z_n + \sum_{k=1}^q\beta_kL^kZ_n = Z_n + \Psi(L)Z_n
        }   
    \end{align*}
    \noindent Where $\Psi(L) = \sum_{k=1}^q \beta_kL^k$ is called the \textbf{moving average operator}.
\end{definition}

\begin{definition}[Autoregressive (AR) model]
    In the $AR(p)$ model of order $p\ge 1$, the current state of the system is expressed as the \textbf{Linear (feedback) combination}:
    \begin{align*}
        \boxed{
            X_n = Z_n + \sum_{k=1}^p \alpha_k X_{n-k}
        }
    \end{align*}

    \noindent Again, we can rewrite the $AR(p)$ model as:
    \begin{align*}
        \boxed{
            X_n = Z_n + \sum_{k=1}^p \alpha_kL^kX_n = Z_n + \phi(L)X_n
        }
    \end{align*}
    \noindent Where the operator $\phi(L) = \sum_{k=1}^p \alpha_kL^k$.
\end{definition}

\begin{proposition}{Recursive solution of $AR(1)$}{recursive_solns_of_ar1}
    The $AR(1)$ process:
    \begin{align*}
        X_n = Z_n + \alpha_1 X_{n-1}
    \end{align*}
    \noindent can be solved recursively in the following cases:
    \begin{itemize}
        \item When $|\alpha_1| < 1\implies$  \textbf{Causal} moving average solution:
        \begin{align*}
            X_n = \sum_{k\ge 0}\alpha_1^k Z_{n-k}, \ n\in\mathbb{Z}
        \end{align*}
        \item When $|\alpha_1| > 1\implies$ \textbf{Non-causal} moving average solution:
        \begin{align*}
            X_n = -\sum_{k\ge 1}\alpha_1^{-k}Z_{n+k}, \ n\in\mathbb{Z}
        \end{align*}
    \end{itemize}

    \noindent\newline With the variance of $X_n$ in both cases are:
    \begin{align*}
        Var(X_n) = \frac{1}{|1 - \alpha_1^2|}
    \end{align*}
    \noindent No such converging solutions exist when $|\alpha_1|=1$.
\end{proposition}

\begin{proof*}[Proposition \ref{prop:recursive_solns_of_ar1}]
    Proving each case, we have
    
    \begin{subproof}{\newline $(i) \ |\alpha_1| < 1$}
        Solving using backward induction, we have:
        \begin{align*}
            X_n &= Z_n + \alpha_1(Z_{n-1} + \alpha_1X_{n-2}) \\
                &= Z_n + \alpha_1(Z_{n-1} + \alpha_1(Z_{n-2} \alpha_1{X_{n-3}})) \\
                &= Z_n + \alpha_1Z_{n-1} + \alpha_1^2Z_{n-2} + \alpha_1^3 X_{n-3} \\
                &\vdots \\
                &= \sum_{k\ge 0} \alpha_1^k Z_{n-k}
        \end{align*}

        \noindent Which converges when the equation $\phi(z)=\alpha_1z=1$ satisfies $|\alpha_1| < 1 \implies |z| > 1$ (Causal).
        \noindent We also have:
        \begin{align*}
            Var(X_n) = \sum_{k\ge0} \alpha_1^{2k}Var(Z_{n-k}) = \sum_{k\ge0}\alpha_1^{2k} = \frac{1}{1-|\alpha_1|^2}
        \end{align*}
    \end{subproof}

    \begin{subproof}{\newline $(ii) \ |\alpha_1| > 1$}
        We can write:
        \begin{align*}
            X_{n+1} = Z_{n+1} + \alpha_1X_n \implies X_n = -\alpha_1^{-1}Z_{n+1} + \alpha_1^{-1}X_{n+1}
        \end{align*}

        \noindent Solving using forward induction, we have:
        \begin{align*}
            X_n &= -\alpha_1^{-1}Z_{n+1} + \alpha_1^{-1}X_{n+1} \\
                &= -\alpha_1^{-1}Z_{n+1} + \alpha_1^{-1}(-\alpha_1^{-1}Z_{n+2} + \alpha_1^{-1}X_{n+2}) \\
                &= -\alpha_1^{-1}Z_{n+1} + \alpha_1^{-1}(-\alpha_1^{-1}Z_{n+2} + \alpha_1^{-1}(-\alpha_1^{-1}Z_{n+3} + \alpha_1^{-1}X_{n+3})) \\
                &= -\alpha_1^{-1}Z_{n+1} - \alpha_1^{-2}Z_{n+2} -\alpha_1^{-3}Z_{n+3}  -\alpha_1^{-3}X_{n+3} \\
                &\vdots \\
                &= -\sum_{k\ge1}\alpha_1^{-k}Z_{n+k}
        \end{align*}

        \noindent Which converges when the equation $\phi(z)=\alpha_1z=1$ satisfies $|\alpha_1| > 1 \implies |z| < 1$ (Non-Causal).
        \noindent We also have:
        \begin{align*}
            Var(X_n) = \sum_{k\ge0} \alpha_1^{-2k}Var(Z_{n-k}) = \sum_{k\ge0}\alpha_1^{-2k} = \frac{1}{|\alpha_1|^2-1}
        \end{align*}
    \end{subproof}
\end{proof*}

\subsubsection{$ARMA$ model}
\begin{definition}[Autoregressive Moving Average (ARMA)  model]
    In the $ARMA(p,q)$ model with orders $p\ge1, q\ge1$, the current state $X_n$ is expressed as the following linear combination:
    \begin{align*}
        \boxed{
            X_n = Z_n + \sum_{k=1}^p \alpha_k X_{n-k} + \sum_{k=1}^q \beta_k Z_{n-k} 
        }
    \end{align*}
    \noindent Making use of the \textbf{moving average operator} $\Psi(L)$ and the $\phi(L)$ operator, we have:
    \begin{align*}
        \boxed{
            X_n =  Z_n + \phi(L)X_n + \Psi(L)Z_n
        }
    \end{align*}
\end{definition}


\subsubsection{$ARIMA$ model}
\begin{definition}[Difference operator]
    Consider the difference operator $\nabla$ defined as:
    \begin{align*}
        \nabla := I - L
    \end{align*}
    \noindent Where $I$ is the identity operator, so that:
    \begin{align*}
        \nabla X_n = X_n - X_{n-1}, \ n\ge1
    \end{align*}
\end{definition}

\begin{proposition}{Iteration of $\nabla$ operator}{iter_of_diff_operator}
    The difference operator can be iterated as followed:
    \begin{align*}
        \nabla^d X_n = \sum_{k=0}^d \begin{pmatrix}
            d \\ k
        \end{pmatrix} (-1)^kX_{n-k}, \ \ d \ge k \ge 0
    \end{align*}
\end{proposition}

\begin{proof*}[Proposition \ref{prop:iter_of_diff_operator}]
    Using the Binomial expansion formula, we have:
    \begin{align*}
        \nabla^d &= (I - L)^d \\
        &= \sum_{k=0}^d I^{d-k}(-L)^k \\
        &= \sum_{k=0}^d (-L)^k = \sum_{k=0}^d (-1)^k L^k \\
        \implies \nabla^dX_n &= \sum_{k=0}^d (-1)^k L^k X_n = \sum_{k=0}^d (-1)^kX_{n-k}
    \end{align*}
\end{proof*}

\begin{proposition}{Recovery of $X_n$ using $\nabla$ operator}{recover_xn_using_diff_operator}
    For any time-series $(X_n)_{n\ge1}$, we can recover $X_n$ using the difference operator via:
    \begin{align*}
        X_n = \sum_{k=0}^d \begin{pmatrix}
            d \\ k
        \end{pmatrix} \nabla^k X_{n-d+k}
    \end{align*}
\end{proposition}

\begin{proof*}[Proposition \ref{prop:recover_xn_using_diff_operator}]
    Applying the Binomial expansion, we have:
    \begin{align*}
        I &= (L + I - L)^d = (L + \nabla)^d \\
            &= \sum_{k=0}^d \begin{pmatrix}
                d \\ k
            \end{pmatrix} L^{d-k} \nabla^k \\
            \implies X_n &= \sum_{k=0}^d \begin{pmatrix}
                d \\ k
            \end{pmatrix} L^{d-k} \nabla^k X_n = \sum_{k=0}^d \begin{pmatrix}
                d \\ k
            \end{pmatrix} \nabla^k X_{n-d+k}
    \end{align*}
\end{proof*}

\begin{definition}[Autoregressive Integrated Moving Average (ARIMA) model]
    In the $ARIMA(p, d, q)$ model, the iterated difference process $(\nabla^dX_n)_{n\ge0}$ is modelled as the $ARMA(p, q)$ time-series:
    \begin{align*}
    \boxed{
        \nabla^d X_n = Z_n + \sum_{k=0}^p \alpha_k \nabla^dX_{n-k} + \sum_{k=0}^q \beta_k Z_{n-k}
    }
    \end{align*}
    \noindent Making use of the \textbf{moving average operator} $\Psi(L)$ and the operator $\phi(L)$, we have:
    \begin{align*}
        \boxed{
            \nabla^d X_n = Z_n + \phi(L)\nabla^dX_n + \Psi(L) Z_n
        }
    \end{align*}
\end{definition}

\subsection{Time-series stationarity}
\subsubsection{Strict \& Weak stationarity}
\begin{definition}[Strict stationarity]
    A time-series $(X_n)_{n\in\mathbb{Z}}$ is \textbf{strictly stationary} if the equality:
    \begin{align*}
        (X_n, X_{n-1}, \dots, X_{n-p}) \simeq (X_{n+m}, X_{n+m-1}, \dots, X_{n+m-p})
    \end{align*}
    \noindent holds \textbf{in distribution} for all $n \in \mathbb{Z}$ and $m, p \ge 0$.
\end{definition}

\begin{definition}[Weak stationarity]
    A time-series $(X_n)_{n\in\mathbb{Z}}$ is \textbf{weakly stationary} if the following holds:
    \begin{itemize}
        \item $(i)$ $\mathbb{E}[X_n] = \mathbb{E}[X_0], \ n\ge0$.
        \item $(ii)$ The auto-covariance:
        \begin{align*}
            (n, m) \to Cov(X_n, X_m)
        \end{align*}
        \noindent depends only on the absolute difference $|n-m|, \ n, m \ge0$. The covariance is defined as $Cov(X, Y) = \mathbb{E}[XY] - \mathbb{E}[X]\mathbb{E}[Y]$.
    \end{itemize}
\end{definition}

\begin{theorem}{Unit root test}{unit_root_test}
    Consider the $AR(p)$ time-series $(X_n)_{n\ge0}$ solution of:
    \begin{align*}
        X_n = Z_n + \phi(L)X_n = Z_n + \alpha_1X_{n-1} + \dots + \alpha_pX_{n-p}
    \end{align*}
    \noindent with the characteristic polynomial:
    \begin{align*}
        \phi(z) = \alpha_1z + \dots + \alpha_qz^q, \ z \in \mathbb{C}
    \end{align*}

    \noindent Then the time-series $(X_n)_{n\ge0}$ is called:
    \begin{itemize}
        \item \textbf{Weakly stationary} : if no solution of $\phi(z) = 1$ \textbf{lies on} the complex unit circle.
        \item \textbf{Causality} : if no solution of $\phi(z) = 1$ \textbf{lies inside} the complex unit circle.
    \end{itemize}

    \noindent The complex unit circle is defined on the complex plane as:
    \begin{align*}
        \{ z \in \mathbb{C} : |z| \le 1 \}
    \end{align*}
\end{theorem}

\subsubsection{Stationarity test}
\begin{definition}[Dickey-Fuller test]
    The Dickey-Fuller test allows us to test the null hypothesis that "The time-series is \textbf{non-stationary}":
    \begin{align*}
        \begin{cases}
            $H_0:|\alpha_1|=1$ $\text{(Non-stationarity)}$
            \\ \\
            $H_1:|\alpha_1|\ne 1$ $\text{(Stationarity)}$ 
        \end{cases}
    \end{align*}
\end{definition}
